Trabalho prático (TP) de {\bfseries Programação e Desenvolvimento de Software II (D\+C\+C204)} da {\bfseries U\+F\+MG} em 2019/2.

Professor\+: Júlio César()

\subsection*{Sistema de Alocação de Demanda de Transporte}

$\vert$ \href{#introdução}{\tt Introdução} $\vert$ \href{#integrantes}{\tt Integrantes} $\vert$ \href{#documentação}{\tt Documentação} $\vert$\href{#slides}{\tt Slides} $\vert$ \href{#user-stories}{\tt User Stories} $\vert$ \href{#como-usar}{\tt Como usar} $\vert$ $\vert$ -\/-\/-\/-\/-\/-\/-\/-\/--- $\vert$ -\/-\/-\/-\/-\/-\/-\/-\/-\/-\/-\/--- $\vert$ -\/-\/-\/-\/-\/-\/-\/-\/-\/-\/-\/--- $\vert$ -\/-\/-\/-\/-\/-\/-\/-\/--- $\vert$ -\/-\/-\/-\/-\/-\/-\/-\/--- $\vert$ -\/-\/-\/-\/-\/-\/-\/-\/--- $\vert$ 



\subsubsection*{Introdução}

Este Trabalho Prático consiste no desenvolvimento de um sistema em linguagem C++ baseado no paradigma de Orientação à Objetos. Para modelar um sistema de alocação de demanda de transporte utilizamos de classes abstratas e conceitos de OO como modularização, polimorfismo, testes de unidade e boas práticas de programação em geral como refatoração, código comentado e versionamento de código.

Desenvolvemos, portanto, um sistema que, dada uma demanda de transporte de uma certa quantidade de determinado produto, para alguma localidade do Brasil ou Exterior, encontra o caminho de menor custo. O custo a ser considerado pode ser monetário ou menor distância ou tempo de percurso, considerando os preços dos serviços, tempo de transporte e capacidade dos diversos modais disponíveis (rodoviário, ferroviário, aéreo e aquaviário).

\subsubsection*{Integrantes}


\begin{DoxyItemize}
\item Estevão de Almeida Vilela ()
\item Wagner Abreu ()
\end{DoxyItemize}

\subsubsection*{Documentação}

Disponível \mbox{[}aqui\mbox{]}() em P\+DF

\subsubsection*{Slides}

Disponível \mbox{[}aqui\mbox{]}() em P\+PT

\subsubsection*{User Stories}

Disponível \mbox{[}aqui\mbox{]}()

\subsubsection*{Como usar}

O usuário do sistema deve inserir a quantidade de carga que deseja transportar entre duas localidades disponíveis. As localidades que estão disponíveis são as capitais do Brasil e algumas capitais de países estrangeiros. 